\documentclass[30pt, a4paper]{article}

% Packages:
\usepackage[
    ignoreheadfoot, % set margins without considering header and footer
    top=2 cm, % seperation between body and page edge from the top
    bottom=2 cm, % seperation between body and page edge from the bottom
    left=2 cm, % seperation between body and page edge from the left
    right=2 cm, % seperation between body and page edge from the right
    footskip=1.0 cm, % seperation between body and footer
    % showframe % for debugging 
]{geometry} % for adjusting page geometry
\usepackage{titlesec} % for customizing section titles
\usepackage{tabularx} % for making tables with fixed width columns
\usepackage{array} % tabularx requires this
\usepackage[dvipsnames]{xcolor} % for coloring text
\definecolor{primaryColor}{RGB}{0, 0, 0} % define primary color
\usepackage{enumitem} % for customizing lists
\usepackage{fontawesome5} % for using icons
\usepackage{amsmath} % for math
\usepackage[
    pdftitle={John Doe's CV},
    pdfauthor={John Doe},
    pdfcreator={LaTeX with RenderCV},
    colorlinks=true,
    urlcolor=primaryColor
]{hyperref} % for links, metadata and bookmarks
\usepackage[pscoord]{eso-pic} % for floating text on the page
\usepackage{calc} % for calculating lengths
\usepackage{bookmark} % for bookmarks
\usepackage{lastpage} % for getting the total number of pages
\usepackage{changepage} % for one column entries (adjustwidth environment)
\usepackage{paracol} % for two and three column entries
\usepackage{ifthen} % for conditional statements
\usepackage{needspace} % for avoiding page brake right after the section title
\usepackage{iftex} % check if engine is pdflatex, xetex or luatex

% Ensure that generate pdf is machine readable/ATS parsable:
\ifPDFTeX
    \input{glyphtounicode}
    \pdfgentounicode=1
    \usepackage[T1]{fontenc}
    \usepackage[utf8]{inputenc}
    \usepackage{lmodern}
\fi

\usepackage{charter}

% Some settings:
\raggedright
\AtBeginEnvironment{adjustwidth}{\partopsep0pt} % remove space before adjustwidth environment
\pagestyle{empty} % no header or footer
\setcounter{secnumdepth}{0} % no section numbering
\setlength{\parindent}{0pt} % no indentation
\setlength{\topskip}{0pt} % no top skip
\setlength{\columnsep}{0.15cm} % set column seperation
\pagenumbering{gobble} % no page numbering

\titleformat{\section}{\needspace{4\baselineskip}\bfseries\large}{}{0pt}{}[\vspace{1pt}\titlerule]

\titlespacing{\section}{
    % left space:
    -1pt
}{
    % top space:
    0.3 cm
}{
    % bottom space:
    0.2 cm
} % section title spacing

\renewcommand\labelitemi{$\vcenter{\hbox{\small$\bullet$}}$} % custom bullet points
\newenvironment{highlights}{
    \begin{itemize}[
        topsep=0.10 cm,
        parsep=0.10 cm,
        partopsep=0pt,
        itemsep=0pt,
        leftmargin=0 cm + 10pt
    ]
}{
    \end{itemize}
} % new environment for highlights


\newenvironment{highlightsforbulletentries}{
    \begin{itemize}[
        topsep=0.10 cm,
        parsep=0.10 cm,
        partopsep=0pt,
        itemsep=0pt,
        leftmargin=10pt
    ]
}{
    \end{itemize}
} % new environment for highlights for bullet entries

\newenvironment{onecolentry}{
    \begin{adjustwidth}{
        0 cm + 0.00001 cm
    }{
        0 cm + 0.00001 cm
    }
}{
    \end{adjustwidth}
} % new environment for one column entries

\newenvironment{twocolentry}[2][]{
    \onecolentry
    \def\secondColumn{#2}
    \setcolumnwidth{\fill, 4.5 cm}
    \begin{paracol}{2}
}{
    \switchcolumn \raggedleft \secondColumn
    \end{paracol}
    \endonecolentry
} % new environment for two column entries

\newenvironment{threecolentry}[3][]{
    \onecolentry
    \def\thirdColumn{#3}
    \setcolumnwidth{, \fill, 4.5 cm}
    \begin{paracol}{3}
    {\raggedright #2} \switchcolumn
}{
    \switchcolumn \raggedleft \thirdColumn
    \end{paracol}
    \endonecolentry
} % new environment for three column entries

\newenvironment{header}{
    \setlength{\topsep}{0pt}\par\kern\topsep\centering\linespread{1.5}
}{
    \par\kern\topsep
} % new environment for the header

\newcommand{\placelastupdatedtext}{% \placetextbox{<horizontal pos>}{<vertical pos>}{<stuff>}
  \AddToShipoutPictureFG*{% Add <stuff> to current page foreground
    \put(
        \LenToUnit{\paperwidth-2 cm-0 cm+0.05cm},
        \LenToUnit{\paperheight-1.0 cm}
    ){\vtop{{\null}\makebox[0pt][c]{
        \small\color{gray}\textit{Last updated in September 2024}\hspace{\widthof{Last updated in September 2024}}
    }}}%
  }%
}%

% save the original href command in a new command:
\let\hrefWithoutArrow\href

% new command for external links:


\begin{document}
    \newcommand{\AND}{\unskip
        \cleaders\copy\ANDbox\hskip\wd\ANDbox
        \ignorespaces
    }
    \newsavebox\ANDbox
    \sbox\ANDbox{$|$}

    \begin{header}
        \fontsize{25 pt}{25 pt}\selectfont Sri Lalitha Pullabhatlapogada

        \vspace{5 pt}

        \normalsize
        \mbox{+1-(510)-513-1969}%
        \kern 5.0 pt%
        \AND%
        \kern 5.0 pt%
        \mbox{\href{mailto:lalithapullabhatlapogada@gmail.com}{lalithapullabhatlapogada@gmail.com}}%
        \kern 5.0 pt%
        \AND%
        \kern 5.0 pt%
        \mbox{\href{https://linkedin.com/in/lalithapullabhatla}{linkedin.com/in/lalithapullabhatla}}%
    \end{header}

    \vspace{5 pt - 0.3 cm}


    \section{Education}



        
        \begin{twocolentry}{
            Oct 2020 – June 2022
        }
            \textbf{National Institute of Technology, Tirichirapalli}
            \end{twocolentry}

        \vspace{0.10 cm}
        \begin{onecolentry}
            \begin{highlights}
                \item Master Of Technology, Computer Science and Engineering
                \item GPA: 9.14/10.0 
                \item \textbf{Coursework:} Computer Architecture, Cloud Computing, Databases, Data Structures and Algorithms, Machine Learning and Deep Learning, Operating Systems
            \end{highlights}
        \end{onecolentry}

        \vspace{0.2 cm}
        
        \begin{twocolentry}{
            Aug 2016 – Sept 2020
        }
            \textbf{Jawaharlal Nehru Technological University, Kakinada}
            \end{twocolentry}

        \vspace{0.10 cm}
        \begin{onecolentry}
            \begin{highlights}
                \item Bachelor Of Technology, Electronics and Communication Engineering
                \item GPA: 7.90/10.0 
                \item \textbf{Coursework:} Computer Networks, Electronic Devices and Circuits, Microprocessors and Microcontrollers
            \end{highlights}
        \end{onecolentry}


    
    \section{Experience}



        
        \begin{twocolentry}{
            Jan 2024 – Nov 2024
        }
            \textbf{Software Engineer 2}, Morgan Stanley -- Bengaluru, India\end{twocolentry}

        \vspace{0.10 cm}
        \begin{onecolentry}
            \begin{highlights}
                \item Focused on integration of the eTrade API with our application to streamline data flows, and implement features such as linking and delinking participant IDs to relevant plans.
                \item Implemented partial and full allocations of grants in two stages at order and block levels, while maintaining the correct sequence of operations.
            \end{highlights}
        \end{onecolentry}


        \vspace{0.2 cm}

        \begin{twocolentry}{
            Jan 2023 – Dec 2023
        }
            \textbf{Software Engineer 1}, Morgan Stanley -- Bengaluru, India\end{twocolentry}

        \vspace{0.10 cm}
        \begin{onecolentry}
            \begin{highlights}
                \item Contributed in reducing administrative effort by developing and implementing an e-filing tool for financial advisors at Morgan Stanley, transitioning the process of filing Form 144 from manual paper-based workflows to a streamlined digital solution.
                \item Enabled users to create multiple 10b5-1 plans simultaneously, improving user efficiency and maintaining compliance.
                \item Ensured that all administrative mandates were accurately replicated across the plans, which necessitated a significant restructuring of the codebase.  
                \item Assisted the team in migrating the application from Angular 11 to Angular 14, ensuring compatibility, improving performance, and leveraging new features of the framework.
            \end{highlights}
        \end{onecolentry}
        
       % \vspace{0.2 cm}
    % \begin{twocolentry}{ Aug 2022 – Dec 2022 }
      %      \textbf{Technology Analyst}, Morgan Stanley -- Bengaluru, India\end{twocolentry}

      %  \vspace{0.10 cm}
       % \begin{onecolentry}
        %    \begin{highlights}
        %        \item Completed comprehensive training covering key topics: Operating systems, Database management, Programming fundamentals, Object-oriented development in C++ and Java, Building distributed systems 
         %       \item Acquired hands-on knowledge through lab sessions and assessments, ensuring practical understanding of core concepts.
          %      \item Developed a small-scale working trading application, demonstrating proficiency in full-stack development.
          %      \item Collaborated with the liquidity team to design and build a dashboard, to present key metrics and insights, enhancing data visualization and decision making.
           % \end{highlights}
        % \end{onecolentry} 

    \section{Projects}
        
        \begin{twocolentry}{
            Aug 2021 – May 2022
        }
            \textbf{Differential Privacy preserving in deep networks}\end{twocolentry}

        \vspace{0.10 cm}
        \begin{onecolentry}
            \begin{highlights}
                \item Developed a framework to employ \begin{math} \epsilon \end{math} - differential privacy by functional mechanism in hybrid deep network so that even with a successful attack, the intruder must not be able to retrieve the identity of the data.
                \item The primary idea is to collaborate with LRP with LSTM by redistributing the Laplacian noise in proportion to the weighted activations, and at a given layer relevance is obtained as a function of activations in lower layer by Deep Taylor decomposition.
                \item Tools Used: Python
            \end{highlights}
        \end{onecolentry}


        \vspace{0.2 cm}

        \begin{twocolentry}{
            Feb 2021 - Apr 2021
        }
            \textbf{Click through rate prediction using FM based neural network}\end{twocolentry}

        \vspace{0.10 cm}
        \begin{onecolentry}
            \begin{highlights}
                \item Trained on Criteo dataset to predict the click through rate using a Factorization-Machine based neural network model.
                \item The model consists of FM component to model second order feature interactions and Deep component to model higher order feature interactions.
                \item Tools used: Python
            \end{highlights}
        \end{onecolentry}
    
    \section{Technologies}
        
        \begin{onecolentry}
            \textbf{Languages:} C, C++, Java, JavaScript, Python, SQL \end{onecolentry}

        \vspace{0.2 cm}

        \begin{onecolentry}
            \textbf{Frameworks:} Angular, Django, Flask, Hibernate, Spring Boot, JUnit, Mockito, Tensorflow
        \end{onecolentry}  

\end{document}